%
% artigo.tex
%
% Workshop de LaTeX do SciELO
%
% Demonstra:
% - Estrutura e capacidades de um documento LaTeX
% - Compilação usando lualatex (e pdflatex)
% - Comandos: \section, \LaTeX, \tableofcontents, \url
% - Classes: article, minimal e seu efeito em \section
% - Erros de compilação
% - Arquivos auxiliares e o comando latexmk -c
%

\documentclass[11pt,a4paper,oneside]{article}
\usepackage{polyglossia}
  \setdefaultlanguage{brazil}
\usepackage{url}

\title{Meu primeiro documento}
\author{João da Silva}

\begin{document}
\frenchspacing

\maketitle

\begin{abstract}
  Neste documento, veremos a estrutura básica e capacidades do \LaTeX.
  Aprenderemos como compilar, o que são comandos simples, comandos que aceitam
  argumentos e como o espaço em branco influencia o resultado final do
  documento. Veremos, também, classes do documento, erros na hora da compilação
  e arquivos auxiliares criados por esse processo.
\end{abstract}

\tableofcontents

\section{Introdução}

Este é o nosso primeiro documento em \LaTeX\ e, para celebrar, faremos uma
lista de coisas que aprendemos:

\begin{itemize}
  \item Estrutura básica de um documento \LaTeX
  \item Compilação e possíveis erros
  \item Comandos
  \item Espaço em branco
  \item Classes
  \item Arquivos auxiliares
\end{itemize}

\section{Mais informações}

Para mais informações e exemplos, veja nosso repositório em
\url{https://gitlab.com/rberaldo/workshop-latex-scielo}.

\begin{center}
  \textit{Tenha um bom dia!}
\end{center}

\end{document}
