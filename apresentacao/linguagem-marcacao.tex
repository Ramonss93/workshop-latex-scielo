% LaTeX: uma linguagem de marcação %%%%%%%%%%%%%%
\begin{frame}[standout]
  \Huge
  \LaTeX: uma linguagem de marcação
\end{frame}

% O LaTeX é uma linguagem de marcação de texto, ou markup
\begin{frame}
  \frametitle{\LaTeX: uma linguagem de marcação}
  \only<1>{\Huge\LaTeX{} é uma linguagem de \emph{markup}}
  \only<2>{\Huge Você \emph{declara} o documento}
  \only<3>{\Huge O programa segue as instruções}
  \only<4>{\Huge Assim como em \textsc{html}, o arquivo fonte é renderizado}
  \only<5>{\Huge Comandos são semânticos}
\end{frame}

% Se os comandos são semânticos, devem ser fáceis de interpretar. O que os
% comandos abaixo significam?
\begin{frame}[fragile]
  \frametitle{\LaTeX: uma linguagem de marcação}
  \begin{minted}[autogobble,fontsize=\huge,breaklines]{latex}
    \section{Introdução}

    \tableofcontents
  \end{minted}
\end{frame}

% Arquivos .tex não contém formatação; são texto plano.
\begin{frame}
  \frametitle{\LaTeX: uma linguagem de marcação}
  \Huge
  \code{.tex} são arquivos de texto plano
\end{frame}
