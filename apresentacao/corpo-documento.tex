% Corpo do documento %%%%%%%%%%%%%%
\section{Corpo do documento}

% O corpo do documento
\begin{frame}[fragile]
  \frametitle{Corpo do documento}
  \begin{minted}[autogobble,fontsize=\LARGE,breaklines]{latex}
    \begin{document}
     …
    \end{document}
  \end{minted}
\end{frame}

% Divisões do documento
\begin{frame}[fragile]
  \frametitle{Corpo do documento: divisões do documento}
  \large
  \begin{multicols}{2}
    \begin{itemize}
      \item\latexcode{\part} (-1)
      \item\latexcode{\chapter} (0)
      \item\latexcode{\section} (1)
      \item\latexcode{\subsection} (2)
      \item\latexcode{\subsubsection} (3)
      \item\latexcode{\paragraph} (4)
      \item\latexcode{\subparagraph} (5)
    \end{itemize}
  \end{multicols}
\end{frame}

% Profundidade
\begin{frame}[fragile]
  \frametitle{Corpo do documento: profundidade das divisões}
  \begin{minted}[autogobble,fontsize=\LARGE,breaklines]{latex}
  \setcounter{secnumdepth}{3}
  \setcounter{tocdepth}{3}
  \end{minted}
\end{frame}

% Comandos estrelados para controlar numeração e o que vai no sumário
\begin{frame}[fragile]
  \frametitle{Corpo do documento: comandos estrelados}
  \begin{minted}[autogobble,fontsize=\large,breaklines]{latex}
  \section*{Esta seção não terá numeração nem aparecerá no sumário}
  \end{minted}
\end{frame}

% Sintaxe para controlar título que vai no sumário
\begin{frame}[fragile]
  \frametitle{Corpo do documento: controlar texto do sumário}
  \begin{minted}[autogobble,fontsize=\large,breaklines]{latex}
  \section[Seção muito longa]{Seção muito longa: provavelmente não ficará muito boa no sumário.}
  \end{minted}
\end{frame}

\begin{frame}
  \frametitle{Corpo do documento: parágrafos}
  \LARGE
  Parágrafos são separados\\
  por linhas em branco
\end{frame}

\begin{frame}[fragile]
  \frametitle{Corpo do documento: espaçamento entre parágrafos}
  \begin{minted}[autogobble,fontsize=\LARGE,breaklines]{latex}
  \setlength{\parskip}{1cm}
  \setlength{\parskip}{1cm plus4mm minus3mm}
  \end{minted}
\end{frame}

\begin{frame}
  \frametitle{Corpo do documento: indentação}
  \LARGE
  Pacote \code{indentfirst}
\end{frame}

\begin{frame}
  \frametitle{Corpo do documento}
  \huge
  Vejamos \filename{exemplo/artigo.tex}
\end{frame}

\begin{frame}
  \frametitle{Corpo do documento}
  \huge
  Resolver \filename{exercicios/meu-artigo.tex}
\end{frame}
