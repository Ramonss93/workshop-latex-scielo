% Pacotes %%%%%%%%%%%%%%
\begin{frame}[standout]
  \Huge
  Pacotes
\end{frame}

% Em exemplos anteriores, a hifenização e algumas strings (como \today) não
% estavam corretas.
\begin{frame}
  \frametitle{Pacotes}
  \huge
  Vimos problemas com localização e hifenização
\end{frame}

% Para resolver, teremos que usar pacotes
\begin{frame}
  \frametitle{Pacotes}
  \Huge
  Solução: pacotes
\end{frame}

% Para carregar pacotes, usamos esta sintaxe
\begin{frame}[fragile]
  \frametitle{Pacotes}
  \begin{minted}[autogobble,fontsize=\LARGE,breaklines]{latex}
    \usepackage[opções]{pacote}
  \end{minted}
\end{frame}

% Para resolvermos nossos problemas, usaremos o pacote polyglossia
\begin{frame}
  \frametitle{Pacotes}
  \Huge
  Pacote \code{polyglossia}
\end{frame}

\begin{frame}
  \frametitle{Pacotes}
  \Huge
  O \code{polyglossia} traz benefícios como:
  \begin{itemize}
    \only<1>{\item Hifenização}
    \only<2>{\item Strings como \mintinline{latex}{\today}}
    \only<3>{\item Convenções tipográficas localizadas}
  \end{itemize}
\end{frame}

\begin{frame}
  \frametitle{Pacotes}
  \Huge
  Como carregar o pacote \code{polyglossia}?
\end{frame}

\begin{frame}[fragile]
  \frametitle{Pacotes}
  \begin{minted}[autogobble,fontsize=\Large,breaklines]{latex}
    \usepackage{polyglossia}
      \setdefaultlanguage{brazil}
  \end{minted}
\end{frame}

% Para encontrar ajuda, podemos ler a documentação oficial dos pacotes que
% estamos usando. Mostrar outras opções do polyglossia, por exemplo.
\begin{frame}
  \frametitle{Pacotes}
  \Huge
  Comprehensive \TeX{} Archive Network

  \url{www.ctan.org}
\end{frame}

\begin{frame}
  \frametitle{Pacotes}
  \huge
  \url{www.ctan.org/pkg/polyglossia}
\end{frame}
