% Símbolos especiais %%%%%%%%%%%%%%
\begin{frame}[standout]
  \Huge
  Símbolos especiais
\end{frame}

% Aspas
\begin{frame}[fragile]
  \frametitle{Aspas}
  \begin{minted}[autogobble,fontsize=\LARGE,breaklines]{latex}
    ``Devemos abrir aspas com dois acentos graves e fechar com duas aspas
    simples.''
  \end{minted}
\end{frame}

% Traços
\begin{frame}[fragile]
  \frametitle{Hífen, travessão e meia-risca}
  \begin{minted}[autogobble,fontsize=\LARGE,breaklines]{latex}
    Leve um guarda-chuva --- ouvi na rádio que pode chover entre 10h--13h.
  \end{minted}
\end{frame}

% Espaços não quebráveis
\begin{frame}[fragile]
  \frametitle{Espaços não quebráveis}
  \begin{minted}[autogobble,fontsize=\LARGE,breaklines]{latex}
    Às 10~horas de ontem…
    Fui à casa do Sr.~Silva…
    Veja mais na página~40.
  \end{minted}
\end{frame}

% Os símbolos a seguir são especiais e devem ser escapados:
\begin{frame}[fragile]
  \frametitle{Caracteres reservados}
  \begin{minted}[autogobble,fontsize=\LARGE,breaklines]{latex}
    # $ % ^ & _ { } ~ \

    \# \$ \% \^{} \& \_ \{ \} \~{} \textbackslash
  \end{minted}
\end{frame}

% Resolver o exercício
\begin{frame}
  \frametitle{Caracteres reservados}
  \huge
  Resolver \filename{exercicios/caracteres\-reservados.tex}
\end{frame}
