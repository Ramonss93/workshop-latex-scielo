% ABNTeX 2 %%%%%%%%%%%%%%
\begin{frame}[standout]
  \Huge
  A classe \code{abntex2}
\end{frame}

\begin{frame}
  \frametitle{A classe \code{abntex2}: apresentação}
  \large
  \begin{quote}
    O abnTeX2, evolução do abnTeX (ABsurd Norms for TeX), é uma suíte para
    LaTeX que atende os requisitos das normas da ABNT (Associação Brasileira de
    Normas Técnicas) para elaboração de documentos técnicos e científicos
    brasileiros, como artigos científicos, relatórios técnicos, trabalhos
    acadêmicos como teses, dissertações, projetos de pesquisa e outros
    documentos do gênero.
  \end{quote}
\end{frame}

% O abnTeX2 implementa muitos novos comandos, que veremos na prática.
\begin{frame}
  \frametitle{A classe \code{abntex2}: comandos e ambientes}
  \LARGE
  Implementa novos comandos:

  \begin{itemize}
    \item\mintinline{latex}{\titulo}
    \item\mintinline{latex}{\autor}
    \item\mintinline{latex}{\imprimircapa}
    \item\mintinline{latex}{citacao} (ambiente)
    \item\mintinline{latex}{resumo} (ambiente)
  \end{itemize}
\end{frame}

\begin{frame}
  \frametitle{A classe \code{abntex2}: implementa diversas normas}
  \LARGE
  Normas regulamentam a organização de textos como trabalhos acadêmicos,
  livros, artigos etc. além de referências e citações
\end{frame}

% Não faremos exercícios, mas o manual é fácil de entender.
\begin{frame}
  \frametitle{A classe \code{abntex2}: documentação}
  \LARGE
  Manual do abnTeX2: \url{www.abntex.net.br}
\end{frame}

% Vejamos um exemplo de documento. Vamos fazer um live coding!
\begin{frame}
  \frametitle{A classe \code{abntex2}: exemplo}
  \huge
  Estudar \filename{exemplos/abntex2/\\trabalho-normatizado.tex}
\end{frame}
